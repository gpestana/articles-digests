\chapter{Background}

\section{Energy consumption in Scientific Computing}
\subsection{Literature review}

% initial explorations of ARM processors for scientific computing
According to  \cite{ACAT13ARM}, the computing requirements for HPC have 
increased particularly in recent
years. Projects of the magnitude and complexity of the Large Hadron Collider are
overwhelming examples of that fact. To achieve results like the discovering of
the Higgs boson and other significant scientific advances, it was necessary a 
 to distribute the processing tasks across several partners and institutions
through the WLCG. The equivalent capacity of such distributed systems was between 
80,000 and 100,000 x86-64 cores in 2012.
Further projects and discoveries will demand even more processing capacity from
the WLCG. For example, as stated by \cite{ACAT13ARM}, to upgrade the LHC 
detectors luminosity to its full power the datasets will increase sizes by 2-3
orders of magnitude and processing power will have to increase in proportion.


% heterogeneous high throughput scientific computing with APM X-gene and Intel Xeon Phi
% TODO intel vs ARM pointers
% 
In \cite{ACAT14ARMDAVID}, a server-purpose ARM machine is compared with the recent
Intel architectures, such as the recent Intel Xeon Phi and a dominating Intel product
intended for HPC workloads (Intel Xeon E5-2650). The workload for comparing
the architectures was ParfullCMS. They based the results on performance (events per
second) and scalability over power (watts). In addition to performance and energy
consumption comparisons, the paper describes the porting endeavors of the CMSSW to
an ARMv8 64-bits architecture.

%TODO describe better different architectures, side by side. 
%TODO describe better experiment setup
In \cite{ACAT14ARMDAVID}, they use an APM X-Gene 1 running on a development board. 
It consists of a 8 physical core processor running at 2.4GHz with 16GB DDR3 memory.
As the authors highlight, the firmware for managing processor ACPI power states was
not yet available when the study was made. Thus, it is expected that the energy 
performance will improve once the firmware is available \cite{ACAT14ARMDAVID}.

Under the circumstances of the experiment, the overall results show that APM X-Gene 
is 2.73 slower than Intel Xeon Phi. From the energy consumption performance (events
per second per watt), the Intel Xeon E-2650 is the most efficient, with APM X-Gene
presenting similar performances despite the absence of platform specific 
optimizations. Therefore, \cite{ACAT14ARMDAVID} concludes by stating that the APM
X-Gene 1 Server-On-Chip ARMv8 64-bit solution is relevant and potentially interesting
platform for heterogeneous high-density computing. 

\section{High Throughput Computing}
\subsection{Literature review}



\section{CERN and the LHC experiment}
\subsection{Literature review}



\section{Energy performance and measurement}
\subsection{Literature review}

% mining questions about software energy consumption
\textbf{on importance of energy consumption for engineers and scientists}
\\
The study conducted by \cite{QUESTIONS_ENERGY}, shows that engineers have been
considering energy consumption as an important factor when developing software.
It consists on an empirical study that aims to understand the opinions and
problems of software developers about energy efficiency. The data that sustain
the conclusions are  mined from
a well-known technical forum (\textit{StackOverflow} \cite{STACKOVERFLOW}).
Although the study is focused in an application-level energy efficiency, it
shows that developers are aware of the importance of energy efficiency in 
computational systems. When trying to understand in depth what questions arise 
more frequently, it is shown that measurement techniques is amongst the most
asked questions by developers. In addition, the study ascertains that the 
\textit{"lack of tool support"} is an important handicap for the development of 
energy efficient software.






\section{ARM architecture}
\subsection{Literature review}


% initial explorations of ARM processors for scientific computing
\textbf{In \cite{ACAT13ARM}:}

- After 2015, processors have hit scaling limits. Two different paths started to be
taken on the processor industry:  development of multiprocessor architectures that
allow to run parallel tasks and the time clock frequency - which have been
increasing throughout the years - stabilized.

- Most High Physics Computing systems run in clusters of several cores.
Additional cores are parallelized and can run at the same time, which allows the
system to scale. However, also commodities such memory, I/O streams and energy
scale proportionally in such architectures.
 



\section{Architecture scheduling based on dynamic energy pricing}

\textbf{In \cite{FUTURE_SMART_GRID}:}

Demand side management are programs implemented by the utility companies to
control and influence the user-side behavior. For example, electrical companies
often fluctuate the energy price depending on the user's demand.

There is need to encourage household owners to *shift* high demand energy
consumptions outside the peak hours, in order to reduce the peak-to-average
(PAR) in load demand.
[ - we aim towards the shifting of schedule different machines  depending on the PAR] 

Direct load control (DLC) gives the utility companies the possibility to
remotely control the household's applications (dim or turn of lights, turn of
thermal equipment, amongst others). Though, this model arises some problems
related with household's privacy
[ - see more  'A direct load control model for virtual power plant management'
  - what if DLC would be implemented for servers and in a heterogeneous
    scheduling scenario? Would it bing any advantage or liability ?]

An alternative to DLC is smart pricing, where users are encouraged to
voluntarily and individually shift their loads out of the peak-hours be
increasing the energy prices when the load is big.

One problem with this approach is synchronization: when a large number of users
shift their peak at the same time for a low-peak time, the PAR may not be
reduced due to the amount of users churning energy at low-peak time.
[ - this might happen as well with our scheduling strategy. If the amount of
users running our scheduling system at the same time is the same, it does not
help to reduce the PAR and the prices might get worst]

The paper suggests that households should synchronize their energy usage and
schedule their energy applications not only according to the price of the energy
at a given time, but also taking into consideration what others are consuming as
well. Thus, by acting in synchronization, the group of users can optimize the
energy the overall energy consumption and its pricing. 

They propose an incentive-based energy consumption pricing model for the smart
grid, where the energy source is shared by several users. The meters communicate
between each other in a distributed network to find the optimal energy
consumption for each user.

Based on game theory, it is shown that through an incentive-based pricing
scheme, an optimal scheduling - where users consume less energy and pay less
money - can be achieved.

\textbf{In \cite{EFF_JOB_SCHEDULING}:}

Because of the magnitude of energy costs in data centers, it is important to
lower the energy consumption in data centers. The servers are composed of
heterogeneous machines from the performance and energy efficiency. In addition,
the data centers may be disposed in different geographical locations and, thus,
have different energy tariffs. The authors of \cite{EFF_JOB_SCHEDULING}, claim that
the key idea to lower the energy bill in data centers is to have energy
efficiency servers and schedule the jobs to where energy is more affordable at a
given time.

In the context of servers distributed over different geographical locations, it
is also important to satisfy fairness and delay constraints. This scenario is
less critical when the server is not distributed, as in our case.

In \cite{EFF_JOB_SCHEDULING}, the authors present an online scheduler that
distributes batch workloads across multiple data centers geographically
distributed. The scheduler aims to minimize the energy consumption of the set of
servers having into consideration fairness and delay requirements.   

The scheduler is inspired on the technique developed by Lyapunov ['Resource
allocation and cross-layer control in wireless networks'] that
optimized time-varying systems. 

The algorithm takes a queue of jobs schedule them to the different servers
having in consideration the (1) server availability, (2) energy price and (3)
job fairness distribution. Consequently, the algorithm is tuned to calculate the
tradeoff between energy pricing, fairness and queueing delay.

- The model:

The data center model takes into consideration the possibility of the energy
prices to vary over time. The state of the data center can be represented at a
given time by a tuple of (i) server availability and (2) energy price.

The job model is characterized by a tuple of (1) service demand - job length -
and (2) the set of data centers the job can be scheduled. 

The scheduler can turn on/off a server when needed. The scheduling is done based
on the server availability and job queue and thus, what matters is the energy
consumed by the server when it is 'idle' or 'busy'.

The scheduler also considers he model fairness (which is not important to
our study, since we focus in a non-distributed server) and queuing delay.
Queuing delay defines the time a job will take to start to be processed,
according to relation of the number of jobs scheduled and machine availability.

In \cite{EFF_JOB_SCHEDULING}, the scheduler developed takes into consideration
the server availability, energy costs, fairness and queuing delay to schedule
random jobs arrivals. It opportunistically schedules jobs when (and to where) 
energy prices are low.

Comparing to our study, though, we do not consider geographically distributed 
servers but rather, we have schedule the jobs based on the heterogeneous set of 
machines existing on the server.



\textbf{In \cite{MIGRATION_CLOUD}:}

This study aims to exploit the temporal and geographical variation of
electricity prices, in the context of data centers. They study algorithms to
schedule (migrate) jobs in data center based on the energy cost and
availability.

When the servers are in different geographical location, costs with data
migration have to be taken into consideration, namely bandwidth costs of moving
the application state and data between data centers. The bandwidth costs
increase proportional to the amount of data migrated between servers.

Their study focuses on inter data center optimization, rather than intra data
center optimization (as our study is aiming for)

The algorithm differs from others in 3 major differences: First, they consider
migration of batches of jobs. Second, the algorithm has into consideration the
future influence of the job scheduling, providing robustness against any future
deviations of the energy price. Finally, they also take into consideration the 
bandwidth costs associated with job migration across data servers.

The main point is to provide a good tradeoff between the energy pricing and the
job migration, taking into consideration the bandwidth prices.

Comparing to our study, we do not approach the problem from an inter data center
perspective, but rather from an intra data center, by scheduling the jobs to
machines depending on their energy performance and the actual energy prices. One
interesting idea from this study that can be used, is the usage of an online
algorithm that takes into consideration the expected prices and also the actual
prices.



--
\textbf{[Notes]}

- Intra data center judicious job scheduling based on the heterogeneous
architecture of the machines.

- Online computation schedule the jobs. Jobs are in a queue in a serial fashion
  and are scheduled depending on the decision of the algorithm at a given time

- There are several studies that aim to leverage the potential of geographical 
  load balancing to provide significant cost savings (see [24, 28, 31, 32, 34,
  39] in \cite{GREENING})


\subsection{Summary}

\begin{comment}
  Add citation in the summary  
\end{comment}

There are two branches of researching on energy efficiency that are related 
with heterogeneous computing and dynamic energy pricing.

In the studies related with the spacial-time dynamic of energy pricing, the 
emphasis is given to the scheduling of jobs across data centers that are located 
in different places. The main idea is to exploit the fact that energy prices 
differ in data centers located in different places. There are concerns with 
fairness, server availability and queue delays. In addition, there are also
several research studies related with migration of cloud computing jobs.

On the heterogeneous computing...

Our solution would present different characteristics from the literature
reviewed so far.

Nevertheless, there are some open points that we might have to consider: whom
and where there will be such heterogeneous intra data center ?
