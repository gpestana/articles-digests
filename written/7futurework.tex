\chapter{Future Work}

There are many new venues for this research work to be improved and further developed. The main areas of research left open are regarding the accuracy and completeness of the experiments while using production hardware; In addition, further development of the algorithm presented for scheduling tasks in a dynamic electricity market. 

\vspace{3mm}

Below, we present a set of suggestions for future work based on the research of this thesis:

\vspace{5mm}

\begin{itemize}
  \item To run more experiments using using only production environments, instead of relying in development boards;

  \item To run more experiments using a methodology where the tools and techniques are as accurate as possible (based on the learning of Chapter 3) and where the results can be compared across all the experiments and to understand if the drawbacks of the approach make it not viable in a real production scenario;

  \item To develop a cross platform and accurate way to perform high resolution power consumption measurements;

  \item To develop accurate mathematical models of energy consumption by data centers consisting of heterogeneous (RISC and CISC) processing nodes;

  \item To develop further the scheduling algorithm for dynamic electricity markets and HTC presented in the Chapter 6 and apply different algorithms that could optimize the solution better than a greedy algorithm;

  \item To improve the energy model used and increase the complexity of the energy profiles in order to the results to be as close to the reality as possible

  \item To test the viability in a real world scenario of the job scheduling in a dynamic electricity market.

\end{itemize}

