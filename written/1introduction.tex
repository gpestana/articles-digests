\chapter{Introduction}

\section{Overview}
%establishing territory
%%claiming centrality
Nowadays, Moore's Law continues to increase the number of transistors per
chipset and the overall technology development at a geometric rate. However, the energy
consumption of the systems have begun to halt the usage of the technology at its
full potential. It is well known that energy efficiency is an important
research topic in computer science, for energy has become a major growth
bottleneck for the systems. In addition, the increasing concerns with energy consumption and
its social, economical and environmental impact in our society has given a
bigger dimension to the discussion.


There are two major approaches to tackle the energy bottleneck in the current
technology panorama. One, is to develop techniques and technologies to better harvest, 
transform and store energy to be used by the systems. This approach aims to provide the
needed energy for technology to reach its full potential. The second path is to improve 
the energy efficiency of the systems. This Thesis focuses on a specific area of
the later approach. 

The concerns with energy consumption and its impact in the current
applications affect industries ranging from mobile devices to big data centers. 
Given the several layers and complexity of the systems nowadays, there are 
considerable number of directions to improve the energy efficiency of the
systems. Throughout this Thesis, we will focus on improving the energy
consumption in High Performance Computing (HPC) applied to Scientific Research.


\subsection{The LHC example}
In some applications, a single computing unit does not have enough resources to
accomplish its tasks. A recurrent strategy is to distribute computational tasks
across a set of computing units that might be spread geographically.

The Large Hadron Collider (LHC) [ref] at the European Laboratory for Particle
Physics (CERN) in Geneva, Switzerland, is an example of a scientific project
whose computing resource requirements are larger that those likely to be provided 
in a single computing unit. Thus, data processing and storage are distributed across 
the Worldwide LHC
Computing Grid (WLCG) [ref], which uses resources from 160 computer centers in 35
countries. Such computational resources have enabled the CMS [ref] and ATLAS [ref]
experiments to discover the Higgs Boson [ref, ref], amongst other scientific 
achievements. 
The WLHC requires a massive amount of computational resources 
(250,000 x86 cores in 2012) and,
proportionally, energy. In the future, with planned increases to the LHC
luminosity [ref], the dataset size will increase by 2-3 orders of magnitude,
posing even more challenges in terms of energy consumption.

The LHC is an example of a massive computational system that needs to improve its
energy efficiency to reach its full potential in the present and future time.
Throughout this Thesis, we will focus primarily on the LHC case. When
appropriate, we will use authentic data and current technology use by the CMS to
study and to draw conclusions with respect to energy efficiency.  


\section{Problem Statement}
%establishing a niche (2)
%%counter-claiming and %%indicating a gap 
A considerable amount of research has been done on leveraging Reduced
Instruction Set Computing (RISC) architectures to minimize energy consumption on 
mobile and energy constrained devices. In such cases, energy consumption is a
priority given the inherent reduced amount of energy available. 

The large quota of ARM architectures in the mobile market supports the fact that
RISC is a good fit for mobile and energy constrained devices.


Similarly to mobile devices, the HPC community has being considering energy
efficiency as a priority in the foreseeable future. However, studies focusing on viability 
of RISC architectures on HPC as a way to minimize energy consumption are not
abundant in the research technology. Furthermore, to the knowledge of the
author, there are no major implementations of such technologies being used in
HPC systems nor in scientific computing.


%%question-raising
It is still unclear whether RISC architectures are a good match to HPC
computing or not. There are open points regarding whether the performance 
constraints of RISC architectures and the high performance requirements of HPC 
workload are acceptable. In addition, it is still unclear if RISC architectures
are more energy efficient under HPC workloads than the conventional Complex 
Instruction Set Computing (CISC) architectures.

%occupying the niche
Therefore, it is of our interest to study the potential impact of RISC architectures in 
the HPC and scientific computing industry. In our opinion, there are two major
lacks that need to be fulfilled: Firstly, there are lack of comparisons between
RISC and CISC architectures under authentic scientific workloads. Secondly,
there are scarce proposal for solutions using RISC in the HPC and scientific
computing.
 

\section{Scope of the Thesis}
%%outlining purposes
The purpose of this Thesis is to answer whether RISC architectures
are a potential fit to HPC and scientific computing from a energy efficiency
perspective. We focus mainly on comparing RISC - most notably ARM chipsets - and
widely used CISC architectures such as Intel processors. For the endeavor, we
use authentic HPC workload from the CMS collider at CERN. 

In order to accomplish the task, we start by investigating the best
and most accurate ways to measure power consumption and compare different
architectures. After, we run several experiments in different chipsets using
authentic workloads from LHC and software used by the CMS team to process the
data generated by the collider. We compare the results and
draw conclusions from them. Finally, based on our learnings, we frame a methodology for
lowering the electrical bill of data centers running under a multi energy
pricing policy, by leveraging the scheduling of machines with different
efficiency profiles.  

   
\section{Contributions}
%%announcing main findings
- Our main findings are ..

\section{Structure of the Thesis}
%%indicating structure of the thesis
This thesis is structured as following. Firstly, we define the context and scope
of the thesis by reviewing relevant and up to date research work. Secondly, we
outline measurements tools and best techniques for energy measurement and
performance in the scientific computing context. Thirdly, we outline the
experiments methodology for comparing the performances of the architectures and 
respective results. In the Chapter 5, we analyze and draw conclusions based on
the results obtained in the experiments. In the Chapter 6, we present some
thoughts on how to lower the energy bill by implementing our learnings thus
far. Finally, we wrap up by outlining possible future work and presenting the
conclusions of this thesis.
