\chapter{Introduction}

\section{Overview}
%establishing territory
%%claiming centrality
Nowadays, computing technology is still developing according to Moore's Law. The number of transistors per
chip-set is still increasing at a geometric rate. However, the energy
consumption of the systems have begun to halt the usage of the technology at its
full potential. Therefore, energy efficiency has became an important
research topic in computer science, for energy has become a major growth
bottleneck in computing systems. In addition, the increasing concerns with energy consumption and
its social, economical and environmental impact in our society has given a
bigger dimension to the discussion.


There are two major approaches to tackle the energy bottleneck in the current
technology panorama. One, is to develop techniques and technologies to better harvest, 
transform and store energy to be used by computing systems. The second approach to the problem is to improve the energy efficiency of the existing computing systems. This thesis focuses primarily on the latter approach. 

The concerns with energy consumption and its impact in the current
applications affect several industries, from mobile device industries to big data centers. 
Given the several layers and complexity of the systems used nowadays, there are 
considerable number of directions that can be taken to improve the overall energy efficiency. Throughout this thesis, we will focus on measuring and improving the energy
consumption in High Throughput Computing (HTC) applied to Scientific Research.


\subsection*{The Large Hadron Collider example}
In some applications, a single computing unit does not have enough resources to
accomplish the required tasks. A recurrent strategy is to distribute computational tasks
across a set of computing centers that might be spread geographically.

The Large Hadron Collider (LHC) \cite{LHC} at the European Laboratory for Particle
Physics (CERN) in Geneva, Switzerland, is an example of a scientific project
whose computing resource requirements are larger that those likely to be provided 
in a single computing unit. Thus, data processing and storage are distributed across 
the Worldwide LHC
Computing Grid (WLCG) \cite{WLCG}, which uses resources from 160 computer centers in 35
countries. Such computational resources have enabled the CMS \cite{CMS} and ATLAS \cite{ATLAS}
experiments to discover the Higgs Boson by S. Chatrchyan, et al. \cite{HIGGS2} and Aad, Georges, et al. \cite{HIGGS1}, amongst other scientific 
achievements. 
The WLHC requires a massive amount of computational resources 
(between 80.00 - 100.00 x86 cores in 2012) and,
proportionally, energy. In the future, with planned increases to the LHC
luminosity, the dataset size will increase by 2-3 orders of magnitude,
posing even more challenges in terms of energy consumption.

The LHC is an example of a massive computational system that needs to improve its
energy efficiency to reach its full potential in the present and future time.
Throughout this thesis, we will focus primarily on the LHC case. When
appropriate, we will use authentic data and current technology used by the CMS.  

\section{Reduced Instruction Set Computing and ARM}
Reduced Instruction Set Computing (RISC) and Complex Instruction Set Computing (CISC) are computing architectures with different computing designs. RISC architectures are designed to operate smaller number of computer instructions than CISC. RISC aims at performing more instructions per second, resulting on more energy performance than CISC the architectures.

The most well known and widely used commercial implementation of CISC architecture are chipsets produced by Intel. On the other hand, ARM has been focusing on the development of RISC based architectures chipsets. Traditionally, ARM chipsets have been used in the energy constrained devices market.

Given the maturity and availability of ARM compared to other RISC computing chipsets, we used ARM chipsets to conduct our research.

\section{Problem Statement}
%establishing a niche (2)
%%counter-claiming and %%indicating a gap 
A considerable amount of research has been done on leveraging RISC architectures to minimize energy consumption on 
mobile and energy constrained devices. In the mobile and energy constrained devices case, it is important to build systems with low energy consumption due to constrains of available power.

The large share of ARM architectures in the mobile market supports the fact that
RISC is a good fit for mobile and energy constrained devices.


Similarly to mobile devices, the High Throughput Computing (HTC) community major concerns with energy efficiency. However, studies focusing on the viability 
of RISC architectures in HTC as a way to minimize energy consumption are not
abundant in the research technology. Furthermore, to the knowledge of the
authors at the time of this research, there are no real world implementation of such technologies in 
HTC data centers.



%%question-raising
It is still unclear whether RISC architectures - and notably ARM chipsets - are a good match to HTC
computing. There are open points regarding whether the performance 
constraints of ARM architecture and the high performance requirements of HTC 
workload are acceptable. In addition, it is still unclear if ARM architecture
is more energy efficient under HTC workloads than the conventional Intel CISC-based architectures.

%occupying the niche
Therefore, it is of our interest to study the potential impact of ARM architectures in 
the HTC and scientific computing industry. In our opinion, there are two major
gaps that need to be filled: Firstly, a comparison between
ARM and Intel architectures under authentic scientific workloads. Secondly,
a solution that leverages RISC architectures under HTC and scientific
computing.
 

\section{Scope of the Thesis}
%%outlining purposes
The purpose of this thesis is to help filling the gaps above mentioned. Firstly, we aim at answering whether RISC architectures
are a potential fit to HTC and scientific computing from a energy efficiency
perspective. We focus mainly on comparing the ARM architecture and Intel, which is a widely used CISC architecture in HTC. During this research, we
use authentic HTC workload and computing frameworks as the CMS experiment. 

Secondly, we use the results of the comparison between ARM and Intel architectures to study how to lower down the electrical bill of HTC data centers running under a dynamic energy
pricing ecosystem.


In order to accomplish the tasks in hand, we start by investigating the best
and most accurate ways to measure power consumption and compare different
architectures. After, we run several experiments in machines with different chip-sets using
authentic workloads and software from the CMS collider. We compare the results and
draw conclusions from them. Finally, based on our learnings, we frame a methodology for
lowering the electrical bill of data centers running under a multi energy
pricing policy, by leveraging the scheduling of machines with different
efficiency profiles.  

   
\section{Contributions}
%%announcing main findings
Our main findings are: 

\begin{itemize}
  \item Accurate techniques and tools for measuring power consumption are important to develop more efficient systems. We researched and outlined best practices for different system levels in the Chapter 3. The results of this research were published in the conference proceedings of the 16th International workshop on Advanced Computing and Analysis Techniques in physics research (ACAT'2014) \cite{ACAT}.

  \item The ARM architecture shows potential for an energy savings in HTC when compared to the x86 systems widely used nowadays. We performed experiments with real world workloads (Chapter 4) and analyzed the results (Chapter 5); 

  \item Heterogeneous computing can be leveraged in HTC in markets where the price of the electricity is dynamic. We have shown how to achieve savings in such environment and developed a scheduling algorithm that accomplishes that (Chapter 6);
\end{itemize}



\section{Structure of the Thesis}
%%indicating structure of the thesis
This thesis is structured as following. In the next chapter, we define the context and scope
of the thesis by synthesizing relevant literature work. In the third chapter, we
outline measurement tools and best techniques for measuring power consumption. The fourth chapter outlines the
experiments methodology used during the different experiment sets. The fifth chapter outlines and analyzes the results based on
the data obtained from the experiments. In the sixth chapter, based on the results of our experiments we present a scheduling algorithm that aims at lowering the energy bill of HTC by using both ARM and Intel architectures. Finally, conclude by outlining possible future work and presenting the
conclusions of this thesis.
