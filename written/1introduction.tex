\chapter{Introduction}


%establishing territory
%%claiming centrality

%%from ACAT's paper
The Large Hadron Collider (LHC) [ref] at the European Laboratory for Particle
Physics (CERN) in Geneva, Switzerland, is an example of a scientific project whose computing resource requirements are larger that those likely to provided in a single computer
center. Data processing and storage are distributed across the Worldwide LHC
Computing Grid (WLCG) [ref], which uses resources from 160 computer centers in 35
countries. Such computational resources have enabled the CMS [ref] and ATLAS [ref]
experiments to discover the Higgs Boson [ref, ref], for example. The WLHC requires a
massive amount of computational resources (250,000 x86 cores in 2012) and,
proportionally, energy. In the future, with planned increases to the LHC
luminosity [ref], the dataset size will increase by 2-3 orders of magnitude,
presenting even more challenges in terms of energy consumption.

%%making topic generalizations 
Energy efficiency is not a trend topic only in High Performance Computing (HPC).
From a  broader perspective, it is known that green computing is an important
research topic in computer science, for energy has become a major growth
bottleneck and resources consumer in technologies ranging from mobile devices to
big datacenters. 
   
%%reviewing previous researches
- Research works that focus on energy consumption and green computing
- Research works that focus on study and developing more efficient computing 
architectures, mostly for mobile computing and energy constrained devices.


%establishing a niche (2)
%%counter-claiming and %%indicating a gap 
A considerable amount of research has been done on leveraging RISC architectures
to minimize energy consumption on mobile and energy constrained devices. Most
notably, that fact is supported by the quota of ARM architectures in the mobile
market.

However, even though green computing has been considered a priority for the 
HPC foreseeable future, studies focusing on viability of reduced instructions set 
architectures on HPC as a mean to minimize energy consumption has not been widely 
discussed amongst the research community. 

%%question-raising
It is still unclear whether RISC architectures are a good match to HPC
computing or not. There are open points regarding whether the performance 
constraints of RISC architectures and the high performance requirements of HPC 
workload are acceptable. In addition, it is still unclear if RISC architectures
are more energy efficient under HPC workloads than the conventional
architectures.

%occupying the niche
Therefore, it is of interest to study the potential impact of RISC architectures in 
the HPC industry. There is lack of comparisons between RISC and CISC processors
under workload characteristic of HPC.

%%outlining purposes
The purpose of this research work is to study the potential of RISC
architecture, namely ARM processors, focusing on authentic HPC workload from
CMS collider at CERN. In order to accomplish the task, we also investigate the best
and more accurate ways to measure power consumption and compare different
architectures. Finally, based on our learnings, we frame a methodology for
lowering the electrical bill of data centers running under a multi energy
pricing policy, by leveraging the scheduling of machines with different
efficiency profiles.  
   
%%announcing main findings
- Our main findings are ..
%%indicating structure of the thesis
- This document is structures as following. 
