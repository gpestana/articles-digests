\chapter{Introduction}


%establishing territory
%%claiming centrality
- Future computational systems will require more computational resources to meet 
its requirements.  \\
- CERN's and CMS's example: present and future \\
%%making topic generalizations 
- Nowadays, it is generally accepted that green computing is an important field
  in data centers \\ 
%%reviewing previous researches
- Research works that focus on energy consumption and green computing
- Research works that focus on study and developing more efficient computing 
architectures, mostly for mobile computing and energy constrained devices.


%establishing a niche (2)
%%counter-claiming and %%indicating a gap 
A considerable amount of research has been done on leveraging RISC architectures
to minimize energy consumption on mobile and energy constrained devices. Most
notably, that fact is supported by the quota of ARM architectures in the mobile
market.

However, even though green computing has been considered a priority for the 
HPC foreseeable future, studies focusing on viability of reduced instructions set 
architectures on HPC as a mean to minimize energy consumption has not been widely 
discussed amongst the research community. 

%%question-raising
It is still unclear whether RISC architectures are a good match to HPC
computing or not. There are open points regarding whether the performance 
constraints of RISC architectures and the high performance requirements of HPC 
workload are acceptable. In addition, it is still unclear if RISC architectures
are more energy efficient under HPC workloads than the conventional
architectures.

%occupying the niche
Therefore, it is of interest to study the potential impact of RISC architectures in 
the HPC industry.


%%outlining purposes
The purpose of this research work is to study the potential of 


%%announcing main findings
%%indicating structure of the thesis
%%evaluating findings
