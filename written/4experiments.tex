\chapter{Experiments}


During our research, we have performed several experiments under different
settings. The main goal was to understand how the ARM and Intel architectures
perform under authentic scientific computing workloads from an energy
consumption standpoint. Furthermore, our goal was to compare the results
obtained to evaluate the potential of ARM architectures to perform HPC tasks, in
comparison to the Intel architectures.
The software used to run or simulate authentic scientific computing tasks is
widely used in production and research at the CMS experience. We have used the
CMSSW framework [ref] (see section Y below) and ParFullCMS [ref] (see section X
below).  

We organized the experiments in 3 sets. The sets differ from methodology,
hardware and software used and general experimental conditions. However, the
main goal of such experiments is always the same: to understand if ARM
architectures have potential to replace Intel in scientific computing from an
energetic standpoint. The techniques and tools used to perform the energy
consumption measurements were based on the study presented on the previous
chapter. The setups of the experiments and tools used to perform the energy
measurements during the experiments are explained and detailed on section W.

The remainder of this chapter is organized as per set of experiment. For each
setup, we present the hardware and software setups and energy measurement tools
used. We will refer the batch of experiments as first (1SE), second (2SE) and
third (3SE) set of experiments throughout the rest of the document. 

\section{First set of experiments}
\section{Second set of experiments}

\begin{figure}[h!]
  \centering
    \includegraphics[width=100mm]{"img/machine_specs"}
    \caption{Specifications of the machines used in the 3SE}
    \label{fig:aalto_quad_clamp}
\end{figure}

\section{Third set of experiments}


