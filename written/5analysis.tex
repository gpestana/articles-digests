\chapter{Analysis}



In this chapter, we present our analysis based on the experiments outlined in the last chapter. The scope of the analysis presented in this section is twofold: to compare the
platforms from an energy efficiency perspective and analyze the tools and techniques used on the different experiment sets. 


The first section and second section of this chapter analyze the different tools and techniques used
to perform the experiments, as well as the results. The first section analyses the first set of experiments, whereas the second section is dedicated to analyses the second set of experiments. As stated before, the difference between first set of experiments and second set of experiments has to do with the machine setups and the tools used to measure the energy consumed. The data and results of the experiments are shown along the analysis.

In the final of this section, we outline the highlights of the analysis for
each set of experiments.

\section{First Set of Experiments}

\begin{figure}[h]
  \centering
    \includegraphics[width=100mm]{"img/aalto/aalto_total_quad"}
    \caption{All stages of the CMSSW experiments on Intel Quad}
    \label{fig:aalto_quad_clamp}
\end{figure}



In figures~\ref{fig:aalto_quad_clamp},~\ref{fig:aalto_atom_clamp} 
and~\ref{fig:aalto_arm_clamp} it is plotted the energy measurements from the beginning until the end of the event generation-simulation by the CMSSW. The energy measurements were done using a meter clamp. The energy measured is represented in the Y-axis and the X-axis represents the time of the experiment in samplings. For each experiment, a sample corresponds to the same time.

In the figures ~\ref{fig:aalto_quad_events}, ~\ref{fig:aalto_atom_events} and ~\ref{fig:aalto_arm_events}, we trimmed out the initialization stage and connection stage of the workload and only show the event processing stage. Whereas the Y-axi represents the energy measured in Watts, the X-axis represents the time spent until the correspondent energy sampling.

Finally, the figures ~\ref{fig:aalto_time} compare the time spent by each of the hardware setups to process the tasks.

\begin{figure}[h]
  \centering
    \includegraphics[width=100mm]{"img/aalto/aalto_total_atom"}
    \caption{All stages of the CMSSW experiments on Intel Atom}
    \label{fig:aalto_atom_clamp}
\end{figure}


\subsubsection*{CMSSW stages}

\begin{figure}[h]
  \centering
    \includegraphics[width=100mm]{"img/aalto/aalto_total_arm"}
    \caption{All stages of the CMSSW experiments on ARM Viridis}
    \label{fig:aalto_arm_clamp}
\end{figure}

Based on the figures~\ref{fig:aalto_quad_clamp}, ~\ref{fig:aalto_atom_clamp} 
and~\ref{fig:aalto_arm_clamp}, we can distinguish three different patterns of energy consumption during the expriement. 
We refer to each pattern as being part of a different CMSSW stage. The stages can be better identified 
when plotting the memory workload and the CPU usage (see ~\ref{fig:memory_stages}).

\begin{figure}[h]
  \centering
    \includegraphics[width=100mm]{"img/aalto/memory_stages"}
    \caption{Memory usage during the 3 different stages}
    \label{fig:memory_stages}
\end{figure}

The first stage consists is the initialization process. During this stage, the memory is the main module being used and thus, it is out of the scope of this work to analyze this stage in depth. 

The second stage is the connection phase. The goal of this stage is to fetch the meta-data from
the CERN servers that allow the event generation-simulation. The meta-data is needed to perform the reconstruction of the events. Once again, 
during this stage the CPU load is low when compared to the memory workload. 

The third stage corresponds to the event processing. This last stage is CPU intensive and it has the most relevant data to to our study, since we our goal is to compare the energy efficiency of the different CPUs. The event processing stage alone is represented by 
the figures ~\ref{fig:aalto_quad_events}, ~\ref{fig:aalto_atom_events} and ~\ref{fig:aalto_arm_events}.


%event processing stage
\begin{figure}[h]
  \centering
    \includegraphics[width=100mm]{"img/aalto/aalto_quadEvents"}
    \caption{CMSSW experiments on Intel Quad - event processing stage}
    \label{fig:aalto_quad_events}
\end{figure}

\begin{figure}[h]
  \centering
    \includegraphics[width=100mm]{"img/aalto/aalto_atomEvents"}
    \caption{CMSSW experiments on Intel Atom - event processing stage}
    \label{fig:aalto_atom_events}
\end{figure}

\begin{figure}[h]
  \centering
    \includegraphics[width=100mm]{"img/aalto/aalto_armEvents"}
    \caption{CMSSW experiments on ARM Viridis - event processing stage}
    \label{fig:aalto_arm_events}
\end{figure}

\subsubsection*{Relative importance of the stages}
The most important stage when studying the energy efficiency of
workload with the CMSSW is the last stage. There are three main reasons for that: Firstly, the CMSSW configuration at CERN has caches that speed up considerably the second stage, thus reducing the energy consumed in the connection stage.
Secondly, the first and second stages are not CPU intensive.
Lastly, the processing stage is the only one that the energy consumption is dependent on the amount of events. Therefore, given any large amount of
data to be processed, the last stage will consume so much more energy than the former stages that the first two stages will become irrelevant in terms of overall energy consumption. Therefore, we focus our energy consumption analysis on the event processing stage only. The event processing stage alone is represented by the figures ~\ref{fig:aalto_quad_events}, ~\ref{fig:aalto_atom_events} and ~\ref{fig:aalto_arm_events}


\subsubsection*{Over-committing CPU and energy efficiency}
We consider a CPU to be over-committed when it has to process more threads or processes than the physical cores available.

If we consider each hardware setup individually, the time needed for running the three stages of the experiment is roughly the same, if the CPU is not over-committed. When
the number of processes exceed the number of available cores, the time to 
process the events increases since there are no available cores to process the
events concurrently. In the over-committed situation, the time increase follows
the ratio \textit{nr\_of\_processes/nr\_of\_cores\_available}. 
For example, if the
number of processes running is 6 and the number of cores available is 4, the
time needed to process the events increases roughly 2/3 compared to when the
CPU is not over-committed.

In terms of energy consumed by the CPU, we do not find any outstanding difference in terms of overall energy efficiency by comparing CPUs that are over-committed vs non over-committed, as we can seen in the Figure ~\ref{fig:aalto_all_results}. However, we expect that if the ratio \textit{nr\_of\_processes/nr\_of\_cores\_available} is large enough, it can affect negatively the energy performance given the energy overhead spent when the jobs are being swapped.


\subsubsection*{Time comparison}
When comparing the time taken by the different architectures to process the same
task (Figure ~\ref{fig:aalto_time}), the pattern is evident. 
Regardless the number of processes, the 
Intel Quad architecture is faster than Intel Atom. In addition, ARM Viridis is faster than Intel Atom.
This fact is due to the architectures characteristics and its specifications, most notably the CPU clock speed.\\

%time comparison
\begin{figure}[h]
  \centering
    \includegraphics[width=100mm]{"img/aalto/aalto_all_time"}
    \caption{Processing time comparison}
    \label{fig:aalto_time}
\end{figure}


\subsubsection*{Energy efficiency comparison}
Given the metrics used in this study (see Metrics section in the Experiments chapter), it is clear that  
systems are  proportionally energy efficient with its ratio performance per 
watts. Therefore, by analyzing the Figure~\ref{fig:aalto_all_results}, it is evident that given the architectures and its configurations, ARM architecture outperforms in terms of energy efficiency its concurrence in all considered 
scenarios. In addition, we conclude that between Intel architectures, Intel Atom is more energy efficient than the Intel Quad. 

%final results
\begin{figure}[h]
  \centering
    \includegraphics[width=150mm]{"img/aalto/aalto_all_results"}
    \caption{Energy efficiency comparison for the first set of experiments - External measurements}
    \label{fig:aalto_all_results}
\end{figure}

%% tools and techniques
\subsubsection*{Measuring tools: external monitoring}
For this set of experiments, the external samples were acquired and recorded 
manually. This factor had a visible impact on the resolution of the measurements. It is clear that the plots show spikes and rough transitions between samples. Moreover, the error
tends to increase proportional to the human interaction with the experiment. 
Therefore, we conclude that it is more effective to use digital and automated ways to sample and
log the data acquired during the measurements.

\subsubsection*{Measuring tools: software-based monitoring}
We used software measurement tools to get an estimated energy consumption by the memory and other system components. In this particular set of experiments, the memory energy measurements done with software were
of particular help to distinguish the different stages, which existence was
unknown before the experiment. The software-based tools can be used as a 
decision support and for learning about unknown and unexpected system behaviors. Thus, even if the output
does not directly show information about energy consumption of the system, it
can be important to support and explain expected - and unexpected - behaviors.
  

\section{Second Set of Experiments}
\subsubsection*{Energy efficiency comparison between Intel Xeon and ARM Odroid}

%ParFullCMS
\begin{figure}[h]
  \centering
    \includegraphics[width=150mm]{"img/acat/results1"}
    \caption{Multithreaded ParFullCMS comparison between Intel Xeon and ARM Odroid}
    \label{fig:parfull_results}
\end{figure}

In the Figure~\ref{fig:parfull_results}, we can see the energy efficiency comparison of Intel Xeon and ARM Odroid. The righmost plot represents the internal energy measurements, whereas the leftmost plot represents the external energy measurements. As in other energy efficiency comparisons in this study, we used the metrics \textit{nr\_of\_events/s/W} to represent the energy performance of the measured systems. 


\begin{figure}[h]
  \centering
    \includegraphics[width=100mm]{"img/aalto/parfull_results_aalto"}
    \caption{Multithreaded ParFullCMS comparison between Intel Xeon, ARM Viridis and ARM Odroid}
    \label{fig:parfull_results_aalto}
\end{figure}

The main conclusion from ~\ref{fig:parfull_results} is that ARM Odroid outperforms Intel Xeon in both internal energy efficiency and external energy efficiency.

\subsubsection*{Energy performance and over-committed CPUs}

It is noticeable that ARM Odroid has a significant energy performance decline when its cores are over-committed. It is also interesting to see that the energy performance decline in the ARM Odroid is relatively larger on the internal energy measurements. One of the reasons we found in our raw results to explain this phenomenon is the large increase of time taken to process the events when the cores are over-committed. Thus, even if the cores are consuming the same power during the event processing stage, the energy efficiency will decrease with the time taken to process the events.  

On the other hand, the energy performance of Intel Xeon does not seem to be significantly affected when over-committed. This can be explained by the fact that Intel Xeon took roughly the same time to process the events when using one core per event and half a core per event.

We believe that the different results between ARM Odroid and Intel Xeon discussed below are due to the fact that ARM Odroid is a development board and it does not implement sophisticated techniques such as Hyper Threading Technology (HTT) by Intel \cite{HTT}. According to Intel, HTT delivers two processing threads per physical core, which allows highly threaded applications to be processed faster. It is expected that if the ratio of  \textit{nr\_of\_threads/core} would be larger than 2, energy efficiency of Intel Xeon would start to decline.  

We believe that if we would over-commit Intel Xeon with more than 4 threads per core, the time to process the workload would increase, which would be followed by a degradation of energy performance.

\subsubsection*{Measurement tools and techniques}

The internal measurement tools used in ARM Odroid and Intel Xeon provide a fine grained resolution to the core level. The TI INA231 and RAPL chips can isolate the pp0, which consists of ALU, FPU, L1 cache and L2 cache when performing energy measurements. 

On the other hand, as stated in \cite{IPMI_resolution}, the lower resolution that IPMI tools offers for internal measurements include energy consumed by the 0P9V, 1P8V, VDD and Vcore rails, which includes the system on the chip, DRAM, Temperature Sensors, and ComboPHY Clock.

As a result of this measurement discrepancy, ~\ref{fig:parfull_results_aalto} shows that ARM Viridis performs worse than any other machine. We believe that this result can be misleading, due to the fact that the tools used to measure the energy consumed by each of the setups measure different components in the CPU. We believe that if components measured in the ARM Viridis would be same as the components measured on the ARM Odroid and Intel Xeon measurements, we would obtain a different result. Namely that ARM Viridis would, at least, perform better that Intel Xeon from an energy consumption perspective. Given the actual setup, we can not scientifically directly compare the results.

\subsection*{Comparison between First set of experiments and Second set of experiments}

When we compare the main results of the first set of experiments (Figure ~\ref{fig:parfull_results_aalto}) and second set of experiments (Figure ~\ref{fig:parfull_results}) we may be inclined to conclude that the setups in the second set of experiments presented an overall more efficiency than the setups the first set of experiments. Again, the used measurement tools play an important role and should not be disregarded when analyzing the results. 
In the first set of experiments, we only performed external measurements. Thus, we discard the possibility to compare the first set of experiments results with the results of the internal measurements of the second set of experiments. 
As for the external measurements performed in both set of experiments, the tools for measuring the energy consumption of both experiments have distinct resolution and grain. In the first set of experiments, we used the clamp meter for measurements in all setups. As for the second set of experiments, we used embedded and computer-assisted tools to perform the external measurements. This discrepancy of tools, its resolutions and errors, make it difficult to compare the results of the first set of experiments and second set of experiments.

However, we can conclude that ARM architecture outperforms the Intel architectures in each and every experiment, regardless the measurement tools and methodologies used.

\section{Conclusions}
The main conclusion of our experiments is that given the setups used in our study, the ARM chip-sets outperform Intel in terms of power efficiency in every experiment.

We learned that the gen-sim mode of CMSSW has different stages and the most relevant from an energy consumption point of view is the latest one, when the events are processed.


In addition, we learned that the tools used to make the experiments play a crucial role in the whole experiment. It is important to assure that the measurement tools and methodologies in use are compatible and suitable to produce results that can be compared. This aspect can be hindered based on the availability of hardware and measurement tools. 
